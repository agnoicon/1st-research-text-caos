
\section{Discussion}

The most relevant conclusions of this chapter are the following:
\begin{enumerate}
\item The two-phase optimisation via simulation of healthcare Emergency
Departments proposed was applied to analyse the administrative strategies
leading to optimum decisions about the physical and human resources
of an ED. In particular, the impact on the economics and the productivity
of Sabadell Hospital ED of different sanitary staff configuration
(v.gr., doctors, triage nurses, admission personnel, emergency nurses,
and x-ray technicians) were analysed.\\

\item The evaluation of the proposal included the \textit{simulation models};
the \textit{decision variables} and\textit{ workloads} used as inputs
of the simulation models; as well as the \textit{metrics} used to
asses the benefits of the proposal. This metrics were defined in term
of three indexes: patient length of stay (LoS) in the ED; number of
attended patients per day (Throughput); and a compound index, the
product of the cost of a given sanitary staff configuration times
patient length of stay (CLoS).\\

\item From interviews with the managers at the EDs of Sabadell hospital
(which provides healthcare services to an average of 160,000 patients/year),
it was found that a basic sanitary of its ED staff is composed by:
9 possible combinations of admission personnel (junior/senior); 9
possible combinations of triage nurses (junior/senior); 5 possible
combinations of emergency nurses (junior/senior); 5 possible combinations
of x-ray technicians (junior/senior); and 14 possible combinations
of doctors (junior/senior) in which a set of examined cases for each
type of staff were analysed as a discrete combinatorial problem.\\

\item In order to analyse the performance of the ED, the real average four
hundred incoming patients that daily arrive to the ED of Sabadell
hospital was divided into four different workload scenarios, up to:
4, 9, 13, and 17 incoming patients hourly, i.e., up to 96, 216, 312,
and 408, respectively for 24hrs.\\

\item All simulations of the ED optimization cases analysed in this work
were carried out in a Linux cluster of the CAOS Department of the
UAB, which has 608 computing cores and 2.2TB of RAM, that is composed
of: 9 nodes of a dual-4 core Intel Xeon E5430, 2.6GHz, 16GB RAM; 1
node of 2xdual-6 core Intel Xeon E5645, 2.4GHz, 24GB RAM; and 8 nodes
of 4x16-cores AMD Opteron ``Interlagos'', 1.66GHz, 256 GB RAM, all
in a switched 1GigE network.\\

\item The evaluation of the proposed methodology aimed to confirm the correct
operation of both the pipeline approach (PA) and the MC plus the K-means
methods, described in \ref{chap3:Math}. To this end, we have first
performed the exhaustive search (ES) to use as baseline method. The
second step of this evaluation consisted on applying the coarse grained
phase, using either the PA, the MC plus K-means methods, or both.
Finally, the fine grained phase was applied in the promising regions
found in the previous step.\\

\item To evaluate the methodology proposed, first the case study A was performed
using the agent-based ED simulator version 1.1. Then the case study
B was performed using the agent-based ED simulator version 1.2. In
both cases, the three metrics and the four different workloads stated
above were tested, and the period simulated was 24 hrs., i.e., one
day of functioning of the ED, in all the experiments.\\

\item After separately applying for cases A and B either the pipeline approach,
PA or the Monte Carlo, MC, plus the K-means methods, or both the \textquotedblleft{}reduced
exhaustive search\textquotedblright{}, the optimum found per each
method for their associated average LoS, average Throughput, and average
CLoS were the same.\\

\item Using the pipeline approach, PA, as the coarse grained phase of the
proposed methodology and then the \textquotedblleft{}reduced exhaustive
search\textquotedblright{} in the promising regions previously found,
our proposal obtained an improvement up to 95.6\% in the computing
time, whereas using the Monte Carlo, MC, plus the K-means methods
and then the \textquotedblleft{}reduced exhaustive search\textquotedblright{}
in the promising regions previously found, our proposal obtained an
improvement up to 72\% in the computing time, both compared with the
exhaustive search used.
\item The optimum solution not always is the best option; it is important
to take into account the sort of optimum when the solutions are going
to be applied into real problems or decision support systems.
\end{enumerate}

\end{document}
